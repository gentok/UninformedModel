%TC:ignore

\clearpage
\appendix
\pagestyle{plain}
\pagenumbering{roman}
\setcounter{page}{1}

\section{Supporting Materials } %(Not Intended for Print)

\par This is the Online Appendix of ``When Uninformed Abstention Improves Democratic Accountability.''

\subsection{Appendix A: Voting Game Proofs}

\subsubsection{The Proof of Lemma 1}

\par In this proof, define $\Pi_{-g}$ as follows:
\begin{align*}
	\Pi_{-g} &= \begin{cases}
		\pi &\text{ if } g = I \\
		1-\pi &\text{ if } g = U
		\end{cases}
\end{align*} 

\par In the voting game, consider the expected payoffs of ideologue voters. 
If approval ideologues ($\beta_g > 1$):
\begin{align*}
EU_{\beta_g > 1 }(v_g=1, x_g=1) &= (1-\Pi_{-g} v_{-g} (1- x_{-g}))(\beta_g + q) + d - c \\
EU_{\beta_g > 1 }(v_g=1, x_g=0) &= \Pi_{-g} v_{-g} x_{-g} (\beta_g + q) - c \\
EU_{\beta_g > 1 }(v_g=0) &= v_{-g} x_{-g} (\beta_g + q)
\end{align*} 
\noindent The above functions imply: 
\begin{align*}
EU_{\beta_g > 1}&(v_g=1, x_g=1) - EU_{\beta_g > 1 }(v_g=1, x_g=1) \\
=& (1- \Pi_{-g} v_{-g})(\beta_g + q) + d > 0 \\
EU_{\beta_g > 1}&(v_g=1, x_g=0) - EU_{\beta_g > 1 }(v_g=0)  \\
%=& (1-\Pi_{-g} v_{-g} (1- x_{-g}) -  v_{-g} x_{-g} )(\beta_g + q) + d -c \notag \\
%=& (1-v_{-g} (\Pi_{-g} (1- x_{-g}) +  x_{-g} )(\beta_g + q) + d -c \notag \\
=& (1-v_{-g} (\Pi_{-g} - x_{-g}(\Pi_{-g} - 1))(\beta_g + q) + d -c > 0 
\end{align*}
\noindent Therefore, approval ideologues prefer $v^*_g=1$ and $x^*_g=1$ regardless of the value of other parameters.

\par If rejection ideologues ($\beta_g<-1$):
\begin{align*}
EU_{\beta_g<-1}(v_g=1, x_g=1) &= (1-\Pi_{-g} v_{-g} (1- x_{-g}))(\beta_g + q)  - c \\
EU_{\beta_g<-1}(v_g=1, x_g=0) &= \Pi_{-g} v_{-g} x_{-g} (\beta_g + q) + d - c \\
EU_{\beta_g<-1}(v_g=0) &= v_{-g} x_{-g} (\beta_g + q)
\end{align*} 
\noindent The above functions imply:
\begin{align*}
EU_{\beta_g<-1}&(v_g=1, x_g=1) - EU_{\beta_g<-1}(v_g=1, x_g=0) \\
=& (1- \Pi_{-g} v_{-g})(\beta_g + q) - d < 0 \\
EU_{\beta_g<-1}&(v_g=1, x_g=0) - EU_{\beta_g<-1}(v_g=0) \\
=& v_{-g} x_{-g} (\Pi_{-g}-1) (\beta_g + q) + d -c  > 0
\end{align*}
\noindent Therefore, rejection ideologues prefer $v^*_g=1$ and $x^*_g=0$ regardless of the values of other parameters.

\hfill $\blacksquare$

\subsubsection{The Proof of Lemma 2}

\par In the voting game, consider the expected payoffs of non-ideologue informed voters ($\beta_g \in [-1,1]$ and know $q$ with certainty). Denote them as $I$ in this proof. $I$   
know the proposed policy quality $q$ and the self-ideology $\beta_I$ with certainty, but uncertain about whether they are pivotal in election (only knows $\pi$). Following equations represent the expected payoffs from possible sets of voting actions:
\begin{align*}
EU_I(v_I=1,x_I=1) = & (1-\pi v_U (1-x_U) ) (\beta_I + q) + d - c \\
EU_I(v_I=1,x_I=0) = &\pi v_U x_U (\beta_I + q) + d - c \\
EU_I(v_I=0) = &v_U  x_U (\beta_I + q)   
\end{align*}
\noindent Voting for approval ($v_I=1,x_I=1$) is more optimal than rejection ($v_I=1,x_I=0$) if and only if:
\begin{align*}
EU_I(v_I=1,x_I=1) &\geq EU_I(v_I=1,x_I=0)  \notag \\
(1 -\pi v_U ) (\beta_I + q) - d &\geq 0 
\end{align*}
\noindent By assumption, $1-\pi v_U \geq 0$ and $d \geq 0$. Also, $I$ prefer approval over rejection if and only if $q=1$.

\par Using the equilibrium vote preference, if $q=1$, $I$ participate if and only if:
\begin{align*}
EU_I(v_I=1,x_I=1) &\geq EU_I(v_I=0) \\
(1- v_U(x_U + (1-x_U) \pi )) (\beta_I + 1) + d - c &\geq 0 
\end{align*}
\noindent By assumption, $\pi - 1 \leq 0$, $v_U \in \{0, 1\}$, $x_U  \in \{0, 1\}$, $d - c > 0$ and $\beta_I + 1 \geq 0$. Therefore, if $q=1$, $I$ participate and vote for approval (i.e., $v_I=1, x_I=1$) regardless of the values of other parameters.  

\par Similarly, if $q=-1$, $I$ participate if and only if:
\begin{align*}
EU_I(v_I=1,x_I=0) &\geq EU_I(v_I=0)  \\
(\pi - 1) v_U x_U (\beta_I - 1) + d - c &\geq 0 
\end{align*}
\noindent By assumption, $\pi - 1 \leq 0$, $v_U \in \{0, 1\}$, $x_U  \in \{0, 1\}$, $d - c > 0$ and $\beta_I - 1 \leq 0$. Therefore, if $q=-1$, $I$ participate and vote for rejection (i.e., $v_I=1, x_I=0$) regardless of the values of other parameters. 

\par It is shown that in the equilibrium, $I$ always choose participation over abstention $v^*_I=1$ and vote for the option aligned with the policy quality $x^*_I=(1+q)/2$. \hfill $\blacksquare$

\subsubsection{The Proof of Lemma 3}

\par In the voting game, consider the expected payoffs of non-ideologue uninformed voters ($\beta_g \in [-1,1]$ and know $q$ only by $\phi$). Denote them as $U$ in this proof. $U$ know the self-ideology $\beta_U$ with certainty, but uncertain about the policy quality, the pivotal voter status (only knows $\pi$), and the ideology of informed voters (only know $\kappa_a$ and $\kappa_r$). Following equations represent the expected payoffs from possible sets of voting actions:
\begin{align*}
EU_U(v_U=1&, x_U=1) = 
\pi (\phi (\beta_U + 1) + (1-\phi) (\beta_U - 1)) + \\
&(1-\pi) (\phi (1-\kappa_{r}) (\beta_U + 1) + (1-\phi) \kappa_{a} (\beta_U - 1)) + \phi d - c \\
EU_U(v_U=1& ,x_U=0) = \notag \\
&(1-\pi) (\phi (1-\kappa_{r}) (\beta_U+1) + (1-\phi) \kappa_{a} (\beta_U - 1))+ (1-\phi) d - c \\
EU_U(v_U=0&) = \phi (1-\kappa_{r}) (\beta_U+1) + (1-\phi) \kappa_{a} (\beta_U - 1)
\end{align*}
\noindent Voting for approval ($v_U=1,x_U=1$) is more optimal than rejection ($v_I=U,x_U=0$) if and only if:
\begin{align*}
EU_U(v_U=1, x_U=1) &\geq EU_U(v_U=1, x_U=0) \notag \\
%\pi (\beta_U + 2\phi - 1) + d (2\phi -1) &\geq 0 \notag \\
\phi &\geq \frac{1}{2} - \frac{\pi \beta_U}{2(\pi + d)} = \phi^*_x
\end{align*}

\par Given the equilibrium approval threshold $\phi^*_x$, consider the participation action $v_U$. When $\phi \geq \phi^*_x$, $U$ choose $v_U=1$ over $v_U=0$ if and only if:
\begin{align*}
EU_U(v_U=1, x_U=1) &\geq EU_U(v_U=0) \notag \\
\phi &\geq \frac{\pi (1- \kappa_{a}) (1-\beta_U) + c}{\pi ((1 - \kappa_{a}) (1-\beta_U) + \kappa_{r} (1+\beta_U)) + d} = \phi^*_{va} 
\end{align*}
\noindent Similarly, when $\phi < \phi^*_x$, $U$ choose $v_U=1$ over $v_U=0$ if and only if: 
\begin{align*}
EU_U(v_U=1, x_U=0) &> EU_U(v_U=0) \notag \\
\phi &< \frac{\pi \kappa_{a} (1-\beta_U) + d - c}{\pi ( \kappa_{a} (1-\beta_U) + (1-\kappa_{r}) (1+\beta_U) ) + d} = \phi^*_{vr} 
\end{align*}

\par Given $\phi^*_x$, $\phi^*_{va}$, and $\phi^*{vr}$, the equilibrium strategy of $U$ $(v^*_U,x^*_U)$ can be written as follows: 
\begin{align*}
(v^*_U, x^*_U) &= 
\begin{cases}
(1, 1) & \text{if and only if $\phi \geq \phi^*_x$ and $\phi \geq \phi^*_{va}$}\\
(1, 0) & \text{if and only if $\phi < \phi^*_x$ and $\phi < \phi^*_{vr}$}\\
(0, 1) & \text{if and only if $\phi \geq \phi^*_x$ and $\phi < \phi^*_{va}$}\\
(0, 0) & \text{if and only if $\phi < \phi^*_x$ and $\phi \geq \phi^*_{vr}$}
\end{cases}
\end{align*}
\hfill $\blacksquare$

\subsubsection{The Proof of Lemma 4}

\par In the voting game, consider the existence of the abstention interval with non-zero width. It exists if the condition satisfies $\phi^*_{v1x0} < \phi^*_x$ or $\phi_{v1x1} > \phi^*_x$. This condition can be represented by the unique threshold in $\pi$. The abstention interval with positive width exists if and only if:
\begin{align*}
\pi &> max \{\pi^*_{v01}, \pi^*_{v02}\}  \text{ or } \pi < min \{\pi^*_{v01}, \pi^*_{v02} \}  \text{ where } \\
\pi^*_{v01} &= \frac{d( d - 2c ) }{\pi (1 - \beta_U^2) (1- \kappa_{r} - \kappa_{a}) - d ( \kappa_{r} (1 + \beta_U) + \kappa_{a} (1-\beta_U )) + 2c}  \\ 
\pi^*_{v02} &= \frac{d (\kappa_{r}(1 + \beta_U) + \kappa_{a} (1-\beta_U)) - 2c}{(1 -\beta_U^2)(1-\kappa_{r}-\kappa_{a})} 
\end{align*}
\noindent The above condition implies that if the width of the abstention interval is positive (i.e., $\lambda(\phi_{v1x0},\phi_{v1x1})>0$), it must be the case that:
$$\phi_{v1x0}=\phi_{vr}<\phi^*_x<\phi_{va}=\phi_{v1x1}$$
\noindent Therefore, if $\lambda(\phi_{v1x0},\phi_{v1x1})>0$:
\begin{align*}
\lambda(\phi_{v1x0},\phi_{v1x1}) = \lambda(\phi_{vr},&\phi_{va}) = \phi_{va}-\phi_{vr}= \\ 
\frac{\pi (1- \kappa_{a}) (1-\beta) + c}{\pi (\kappa_{r} (1+\beta_U) + (1 - \kappa_{a}) (1-\beta_U)) + d} &- \frac{\pi \kappa_{a} (1-\beta) + d - c}{\pi ( (1-\kappa_{r}) (1+\beta_U) + \kappa_{a} (1-\beta_U)) + d} \\
\end{align*}

\par Consider the role of $c$. $\phi_{va}$ is increasing in $c$ and $\phi_{vr}$ is decreasing in $c$. Therefore, $\lambda(\phi_{v1x0},\phi_{v1x1})$ is increasing in $c$. 

\par Consider the role of $d$. The denominator of $\phi_{va}$ is strictly increasing in $d$, thus $\phi_{vr}$ is decreasing in $d$. For $\phi_{vr}$, the following statements hold:
\begin{align*}
\frac{\pi \kappa_{a} (1-\beta) - c}{\pi ( (1-\kappa_{r}) (1+\beta_U) + \kappa_{a} (1-\beta_U))} &< 1 \text{ and } d/d = 1 \Rightarrow \lim_{d \to \infty} \phi_{vr} = 1 
\end{align*}
\noindent Therefore, $\phi_{va}$ is decreasing in $d$ and $\phi_{vr}$ is increasing in $d$: $\lambda(\phi_{v1x0},\phi_{v1x1})$ is decreasing in $d$.

\par Consider the role of $\kappa_{r}$. $\phi_{vr}$ is weakly increasing in $\kappa_{r}$ and $\phi_{va}$ is weakly decreasing in $\kappa_{r}$. Therefore, the abstention interval $\lambda (\phi^*_{v1x0}, \phi^*_{v1x1})$ is weakly decreasing in $\kappa_{r}$. 

\par Consider the role of $\kappa_{a}$. $\pi \kappa_{a} (1-\beta_U)= k_{0a}$ is weakly increasing in  $\kappa_{a}$  and $\pi (1-\kappa_{a}) (1-\beta_U) = k_{0b}$ is weakly decreasing in $\kappa_{a}$. Also, $k_{0a} \geq 0$ and $k_{0b} \geq 0$. Then, following equations represent the partial derivative of $\phi_{va}$ in terms of $k_{0b}$ and the partial derivative of $\phi_{vr}$ in terms of $k_{0a}$:
\begin{align*}
%\lambda (\phi^*_x, \phi^*_{va}) &= \frac{k_{0b} + c}{k_{0b} + \pi \kappa_{r}(1+\beta_U) + d} - \left( \frac{1}{2} - \frac{\pi \beta_U}{2(\pi + d)} \right) \\
\frac{\partial}{\partial k_{0b}} \lambda (\phi^*_x, \phi^*_{va}) &= \frac{  \pi \kappa_{r}(1+\beta_U) + d - c}{(k_{0b} + \pi \kappa_{r}(1+\beta_U) + d)^2} \\
%\lambda (\phi^*_{vr}, \phi^*_x) &= \left( \frac{1}{2} - \frac{\pi \beta_U}{2(\pi + d)} \right) - \frac{k_{0a} + d - c}{k_{0a} + \pi (1-\kappa_{r}) (1 + \beta_U) + d} \\  
\frac{\partial}{\partial k_{0a}} \lambda (\phi^*_{vr}, \phi^*_x) &=  - \frac{\pi (1-\kappa_{r}) (1 + \beta_U) + 2 d - c}{(k_{0a} + \pi (1-\kappa_{r}) (1+\beta_U) + d)^2} 
\end{align*}
\noindent Since $d > c \geq 0$, $\pi \in [0,1]$, and $\kappa_{r} \in [0,1)$, $\frac{\partial}{\partial k_{0b}} \lambda (\phi^*_x, \phi^*_{vapp}) $ is strictly positive and $ \frac{\partial}{\partial k_{0b}} (\phi^*_{vrej}, \phi^*_x)$ is strictly negative. Therefore, $\lambda (\phi^*_x, \phi^*_{vapp})$ is strictly increasing in $k_{0b}$ and $\lambda (\phi^*_{vrej}, \phi^*_x)$ is strictly decreasing in $k_{0a}$. Consequently, the abstention interval $\lambda (\phi^*_{v1x0}, \phi^*_{v1x1})$ is weakly decreasing in $\kappa_{a}$. 

\hfill $\blacksquare$

\subsubsection{The Proof of Proposition 1}

\par In the voting game, consider the relationship between $\lambda(\phi_{v1x0},\phi_{v1x1})$ and $\pi$. From Lemma 4, $\lambda(\phi_{v1x0},\phi_{v1x1})>0$ implies $\lambda(\phi_{v1x0},\phi_{v1x1})=\lambda(\phi_{vr},\phi_{va})$. Then, take the partial derivative of $\phi_{vr}$ in terms of $\pi$:
\begin{align*}
\frac{\partial}{\partial \pi} \phi_{vr} = \frac{-(d-c)(1-\kappa_r)(1+\beta_U) + c\kappa_a(1-\beta_U)}{\left(\pi\left(\left(1-\kappa_r\right)\left(1+\beta_U\right)+\kappa_a\left(1-\beta_U\right)\right)+d\right)^2}
\end{align*}
\noindent By assumption, the denominator of $\frac{\partial}{\partial \pi} \phi_{vr}$ is larger than zero. Therefore, $\phi_{vr}$ is increasing in $\pi$ if and only if:
\begin{align*}
-(d-c)(1-\kappa_r)(1+\beta_U) + c\kappa_a(1-\beta_U) &> 0 \\
\frac{\kappa_a(1-\beta_U)}{(1-\kappa_r)(1+\beta_U)} &> \frac{d}{c} - 1
\end{align*}

\par Take the partial derivative of $\phi_{va}$ in terms of $\pi$:
\begin{align*}
\frac{\partial}{\partial \pi} \phi_{vr} = \frac{(d-c)(1-\beta_U)(1-\kappa_a) - c\kappa_r(1+\beta_U)}{\left(\pi\left(\kappa_r\left(1+\beta_U\right)+\left(1-\kappa_a\right)\left(1-\beta_U\right)\right)+d\right)^2}
\end{align*}
\noindent By assumption, the denominator of $\frac{\partial}{\partial \pi} \phi_{va}$ is larger than zero. Therefore, $\phi_{va}$ is decreasing in $\pi$ if and only if:
\begin{align*}
(d-c)(1-\beta_U)(1-\kappa_a) - c\kappa_r(1+\beta_U) &< 0 \\
\frac{\kappa_r(1+\beta_U)}{(1-\kappa_a)(1-\beta_U)} &> \frac{d}{c} - 1
\end{align*}
\hfill $\blacksquare$

\subsubsection{The Proof of Proposition 2}

\par In the voting game, consider the case where $c=0$. From the proof of Proposition 1:
\begin{align*}
\lim_{c \to_{-} 0} d/c - 1 = \infty > \frac{\kappa_a(1-\beta_U)}{(1-\kappa_r)(1+\beta_U)} > \frac{\kappa_r(1+\beta_U)}{(1-\kappa_a)(1-\beta_U)}
\end{align*}
\noindent Therefore, the only possible form of abstention under $c=0$ is delegatory abstention. 

\par From the proof of Lemma 4, $c=0$ implies that $\pi^*_{v02}>0$. Additionally, $c=0$ implies that $d > 2c$, indicates that the numerator of $\pi^*_{v01}$ is a positive value. Then, the following statements hold:
\begin{align*}
\pi<\pi^*_{v02} &\Rightarrow \text{ the denominator of $\pi^*_{v01} < 0$} \\
\pi<\pi^*_{v02} &\Rightarrow \pi^*_{v01} < 0 \Rightarrow \pi < \pi^*_{v01} \text{ does not exist} \\
\pi>\pi^*_{v02} &\Rightarrow \text{ the denominator of $\pi^*_{v01} > 0$} \\
\pi>\pi^*_{v02} &\Rightarrow \pi^*_{v01} > 0 \Rightarrow \pi > \pi^*_{v01} \text{ may exist} \\
\end{align*}
\noindent From the above statements, the non-zero delegatory abstention interval (i.e., $\lambda(\phi^*_{v1x0},\phi^*_{v1x1}$) exists if and only if $\pi > max\{\pi^*_{v01},\pi^*_{v02}\}$.

\par Consider the existence of $\pi > \pi^*_{v01}$. Since $\pi \leq 1$ by definition, this condition implies that $\pi^*_{v01} < 1$. $\pi^*_{v01}$ is decreasing in $\pi$ and increasing in $\kappa_a$, $\kappa_r$. Also, $c=0 \Rightarrow d - 2c > 0$ implies that $\pi^*_{v01}$ is increasing in $d$. The above relationships suggest that if $\pi > \pi^*_{v01}$ holds under $c=0$, it holds for sufficiently high $\pi$ and sufficiently low $\kappa_a$, $\kappa_r$, and $d$.

\par To check the existence of such $\pi^*_{v01} < 1$, set $\pi$ to the maximum value $1$ and set $\kappa_a$ and $\kappa_r$ to the minimum value $0$. Under this condition, the non-zero abstention interval exists if and only if:
\begin{align*}
\pi^*_{v01}[\pi=1,\kappa_a=0,\kappa_r=0,c=0] = \frac{d^2 }{(1 - \beta_U^2)} &< 1\\
d^2 &< 1 - \beta_U^2 \\
d &< +\sqrt{1 - \beta_U^2} \text{ (since $d>0$)} 
\end{align*}
\noindent By assumption, $d < +\sqrt{1 - \beta_U^2}$ exists for any $\beta_U \in (-1,1)$. 

\par Consider the existence of $\pi > \pi^*_{v02}$. This condition implies that $\pi^*_{v02} <1$. When $c=0$, $\pi^*_{v02}$ is increasing in $\kappa_a$, $\kappa_r$, and $d$. To check the existence of such $\pi^*_{v02} <1$, fix $\kappa_a$ and $\kappa_r$ to $0$. Then, $\pi^*_{v02}[\kappa_a=0,\kappa_r=0,c=0] = 0 < 1$.

\hfill $\blacksquare$ 

\subsubsection{The Proof of Proposition 3}

\par In the voting game, consider the expected policy utility (EPU, $E[a(\beta_g + q)]$) of non-ideologue uninformed voters. EPU is increasing in the abstention if and only if the expected policy utility from the electoral decision dominated by informed voters ($EPU_U[\pi=0]$) exceeds the expected utility from the uninformed vote ($EPU_U[\pi=1,v_U=1]$). First, assume that $\phi \geq \phi^*_x$ and the abstention interval exists. Compare EPUs from the informed vote and the uninformed approval vote:  
\begin{align*}
EPU_U[\pi=0] &\geq EPU_U[\pi=1,v_U=1,x_U=1] \\
\phi(1-\kappa_r)(\beta_U+1) + (1-\phi)\kappa_a(\beta_U-1) &\geq \phi(\beta_U+1) + (1-\phi)(\beta_U-1) \\
\phi &\leq \frac{(1-\kappa_a)(1-\beta_U)}{(1-\kappa_a)(1-\beta_U)+\kappa_r(1+\beta_U)}
\end{align*}
\noindent From Lemma 3, non-ideologue uninformed voters abstain for $\phi$ lower than $\phi^*_{v1x1}$. Under the delegatory abstention context, this threshold is maximized at $\pi=1$: 
\begin{align*}
max(\phi^*_{v1x1}) = \frac{(1-\kappa_a)(1-\beta_U) + c}{(1-\kappa_a)(1-\beta_U)+\kappa_r(1+\beta_U) + d}
\end{align*}
From Proposition 1 the following statement holds under the delegatory context:
\begin{align*}
\frac{\kappa_r(1+\beta_U)}{(1-\kappa_a)(1-\beta_U)} &\leq \frac{d}{c} - 1\\
\frac{c}{d} &\leq \frac{(1-\kappa_a)(1-\beta_U)}{(1-\kappa_a)(1-\beta_U)+\kappa_r(1+\beta_U)}
\end{align*}
The above inequality implies:  
\begin{align*}
max(\phi^*_{v1x1}) \leq \frac{(1-\kappa_a)(1-\beta_U)}{(1-\kappa_a)(1-\beta_U)+\kappa_r(1+\beta_U)}
\end{align*}
\noindent Therefore, for all $\phi$ uninformed voters abstain under the delegatory abstention context, $EPU_U[\pi=0] \geq EPU_U[\pi=1,v_U=1,x_U=1]$ holds. The expected policy utility of non-ideologue uninformed voters is increasing in the delegatory abstention if $\phi \geq \phi^*_x$. 

\par On the other hand, under the discouraged or mixed abstention context (with $\phi^*_{v1x1}$ weakly decreasing in $\pi$), $\phi^*_{v1x1}$ is minimized at $\pi=1$.
\begin{align*}
min(\phi^*_{v1x1}) = \frac{(1-\kappa_a)(1-\beta_U)+c}{(1-\kappa_a)(1-\beta_U)+\kappa_r(1+\beta_U)+d}
\end{align*}
\noindent Proposition 1 implies:
\begin{align*}
\frac{\kappa_r(1+\beta_U)}{(1-\kappa_a)(1-\beta_U)} &> \frac{d}{c} - 1\\
\frac{c}{d} &> \frac{(1-\kappa_a)(1-\beta_U)}{(1-\kappa_a)(1-\beta_U)+\kappa_r(1+\beta_U)}
\end{align*}
\noindent Therefore, $EPU_U[\pi=0] \geq EPU_U[\pi=1,v_U=1,x_U=1]$ does not hold for some $\phi$ uninformed voters abstain under the discouraged or mixed abstention context (with $\phi^*_{v1x1}$ weakly decreasing in $\pi$). Under those contexts (when $\phi \geq \phi^*_x$), the expected policy utility of non-ideologue uninformed voters is potentially decreasing in the abstention.

\par Second, assume that $\phi < \phi^*_x$ and the abstention interval exists. Compare EPUs from the informed vote and the uninformed rejection vote.  
\begin{align*}
EPU_U[\pi=0] &\geq EPU_U[\pi=1,v_U=1,x_U=0] \\
\phi(1-\kappa_r)(\beta_U+1) + (1-\phi)\kappa_a(\beta_U-1) &\geq 0 \\
\phi &\geq \frac{\kappa_a(1-\beta_U)}{\kappa_a(1-\beta_U)+(1-\kappa_r)(1+\beta_U)}
\end{align*}
\noindent From Lemma 3, non-ideologue uninformed voters abstain for $\phi$ higher than $\phi^*_{v1x0}$. Under the delegatory abstention context, this threshold is minimized at $\pi=1$: 
\begin{align*}
min(\phi^*_{v1x0}) = \frac{\kappa_a(1-\beta_U)+d-c}{\kappa_a(1-\beta_U)+(1-\kappa_r)(1+\beta_U)+d}
\end{align*}
From Proposition 1 the following statement holds under delegatory context: 
\begin{align*}
\frac{\kappa_a(1-\beta_U)}{(1-\kappa_r)(1+\beta_U)} &\leq \frac{d}{c} - 1 \\
\frac{d-c}{d} &\geq \frac{\kappa_a(1-\beta_U)}{\kappa_a(1-\beta_U)+(1-\kappa_r)(1+\beta_U)}
\end{align*}
The above inequality implies:  
\begin{align*}
min(\phi^*_{v1x0}) \geq \frac{\kappa_a(1-\beta_U)}{\kappa_a(1-\beta_U)+(1-\kappa_r)(1+\beta_U)}
\end{align*}
\noindent Therefore, for all $\phi$ uninformed voters abstain under the delegatory abstention context, $EPU_U[\pi=0] \geq EPU_U[\pi=1,v_U=1,x_U=0]$ holds. The expected policy utility of non-ideologue uninformed voters is increasing in the delegatory abstention if $\phi < \phi^*_x$. 

\par On the other hand, under the discouraged or mixed abstention context (with $\phi^*_{v1x0}$ weakly increasing in $\pi$), $\phi^*_{v1x0}$ is maximized at $\pi=1$.
\begin{align*}
max(\phi^*_{v1x0}) = \frac{\kappa_a(1-\beta_U)+d-c}{\kappa_a(1-\beta_U)+(1-\kappa_r)(1+\beta_U)+d}
\end{align*}
\noindent Proposition 1 implies:
\begin{align*}
\frac{\kappa_a(1-\beta_U)}{(1-\kappa_r)(1+\beta_U)} &> \frac{d}{c} - 1 \\
\frac{d-c}{d} &< \frac{\kappa_a(1-\beta_U)}{\kappa_a(1-\beta_U)+(1-\kappa_r)(1+\beta_U)}
\end{align*}
\noindent Therefore, $EPU_U[\pi=0] \geq EPU_U[\pi=1,v_U=1,x_U=1]$ does not hold for some $\phi$ uninformed voters abstain under the discouraged or mixed abstention context (with $\phi^*_{v1x0}$ weakly decreasing in $\pi$). Under those contexts (when $\phi < \phi^*_x$), the expected policy utility of non-ideologue uninformed voters is potentially decreasing in the abstention.

\hfill $\blacksquare$

\clearpage
\subsection{Appendix B: Accountability Game Proofs}

\subsubsection{The Proof of Lemma 5}

\par In the accountability game, the following equation represents the payoff function of the low-capacity policymaker.
\begin{align*}
u_P[T=L] = \begin{cases}
2 - (1+q) &\text{ if the policy is approved} \\
- (1+q) \cdot \cfrac{1+q}{2} &\text{ if the policy is rejected} \\
\end{cases}
\end{align*}
Then, the following statements stand for the the expected utilities from policymaking:
\begin{align*}
EU_{P_L}[q_L=1] &= 2 \cdot Pr(approval)[q_L=1]  - 2 = 2 (Pr(approval)[q=1]-1) \leq 0\\
EU_{P_L}[q_L=-1] &= 2 \cdot Pr(approval)[q_L=-1] \geq 0 \\
EU_{P_L}[q_L=1] &\leq 0 \leq EU_{P_L}[q_L=-1]
\end{align*}
\noindent Therefore, assuming that players never play weakly dominated strategies, the low-capacity policymaker ($T=L$) always proposes a low-quality policy ($q=-1$) in the equilibrium. 
\hfill $\blacksquare$

\subsubsection{The Proof of Proposition 4}

\par In the accountability game, the low-capacity policymaker ($P_L$) chooses a low-quality policy for sure (Lemma 5) and voters use the logic explained in Lemma 1, 2 and 3 to determine their behavior. Given that, consider the decision of the high-capacity policymaker ($P_H$) under $\pi=0$ (pivotal voters are always informed). Following function illustrates the expected utility:
\begin{align*}
EU_{P_H} &= 
\begin{cases}
(1-\kappa_{r}) 2 - 1  & \text{if $q=1$} \\
\kappa_{a} 2  & \text{if $q=-1$} 
\end{cases} 
\end{align*}
$P_H$ is is better off proposing $q=1$ than $q=-1$ if and only if:
\begin{align*}
EU_{P_H}[q=1] &> EU_P[q=-1] \\
\kappa_{a} + \kappa_{r} < 0.5
\end{align*}
Therefore, $P_H$ best responds by $q=1$ if $\kappa_{a} + \kappa_{r} < 0.5$ and by $q=-1$ if $\kappa_{a} + \kappa_{r} \geq 0.5$. 
\hfill $\blacksquare$

\subsubsection{The Example that $\phi^*_H$ moves down from $1$ under $\kappa_{a} + \kappa_{r} < 0.5$}

\par Suppose that $\kappa_a+\kappa_r<0.5$. Under this condition, $P_H$ proposes a high-quality policy for sure (i.e., $\phi^*_H=1$) to the fully informed pivotal voters ($\pi=0$, Proposition 4). Since $\phi^*_H=1$ is the maximum possible value of $\phi^*_H$, showing the existence of $\phi^*_H<1$ under $\pi>0$ is enough to show that $\phi^*_H$ is weakly decreasing in the positive probability of uninformed pivotal voters. 

\par Consider the condition where $p \geq \phi^*_{v1x1}[\pi=0]$ and $\phi^*_{v1x1}$ decreasing in $\pi$ (see the discouraged abstention described in relation to Proposition 1). This condition implies that $p \geq \phi^*_{v1x1}$ for any $\pi$. Also, $EU_{P_H}[v_U=1,x_U=1]$ is increasing in $\phi_H$ (i.e., maximized at $\phi^*_H=1$) if and only if:
\begin{align*}
1-2(\kappa_a+\kappa_r)-2\pi(1-\kappa_a-\kappa_r)&>0\\
\pi &< \frac{1-2(\kappa_a+\kappa_r)}{2-2(\kappa_a+\kappa_r)}
\end{align*}
\noindent By assumption $\kappa_a+\kappa_r<0.5$, the right-hand side of the above inequality is larger than $0$. Consequently, the inequality always holds for $\pi=0$. On the other hand, when $\pi>0$, there exists a condition where $EU_{P_H}[v_U=1,x_U=1]$ is decreasing in $\phi_H$ for sufficiently high $\pi$. This example illustrates the existence of $\phi^*_H<1$ under $\pi=0$: $\phi^*_H$ is weakly decreasing in the presence of potentially pivotal uninformed voters ($\pi>0$). 

\subsubsection{The Proof of Proposition 5}

\par In the accountability game, consider the decision of the high-capacity policymaker ($P_H$) under the positive prior probability of uninformed pivotal voters ($\pi>0$). In this game, uninformed voters do not observe $\phi$ but do know the prior probability of the high-capacity policymaker ($p$). Suppose that $P_H$ chooses the policy according to the prior probability of a high-quality policy $\phi_H$. The choice of the prior probability is the mixed strategy of $P_H$. Since the low-capacity policymaker never proposes a high-quality policy, the prior probability of a high-quality policy for voters is $\phi = p \cdot \phi_H$. 

\par Conditional on the behavior of the non-ideologue uninformed voters, following three equations represent the expected utility of $P_H$ under $\pi > 0$:
\begin{align*}
EU_{P_H}[v_U = 0] =& (1-\pi (\kappa_{a} + \kappa_{r})) (2 \kappa_{a} + \phi_H (1- 2(\kappa_{a} + \kappa_{r}))) \\ &+\pi (2 \kappa_{a} - \phi_H (\kappa_{a} + \kappa_{r}))  \\
EU_{P_H}[v_U = 1, x_U = 1] =& (1-\pi) (2 \kappa_{a} + \phi_H (1- 2(\kappa_{a} + \kappa_{r})))  \\ &+ \pi (2 \kappa_{a} - \phi_H + 2 (1- (\kappa_{a} + \kappa_{r}))  \\
EU_{P_H}[v_U = 1, x_U = 0] =& (1-\pi) (2 \kappa_{a} + \phi_H (1- 2(\kappa_{a} + \kappa_{r})))  \\ &+ \pi (2 \kappa_{a} - \phi_H) 
\end{align*}

\par Suppose that $\kappa_a + \kappa_r \geq 0.5$.  If $\pi=0$, $P_H$ chooses a low-quality policy for sure (i.e., $\phi^*_H=0 \Rightarrow \phi^* = p \cdot \phi^*_H = 0$). Non-ideologue uninformed voters choose rejection (i.e., $v_U=1$, $x_U=0$) for sure (Lemma 3). Under this condition, the expected utility of $P_H$ can be calculated as follows:
\begin{align*}
EU_{P_H}[\phi_H=0, v_U=1, x_U=0] = 2 \kappa_a
\end{align*}
\noindent If $\pi>0$, It may be possible for $P_H$ to influence the decision of non-ideologue uninformed voters by setting $\phi_H>0$. Following statements describe when and how $P_H$ can influence the decision of non-ideologue uninformed voters. Since the expected cost of policymaking is increasing in $\phi_H$, $P_H$ chooses the lowest possible value of $\phi_H$ that can ensure the particular behavior of uninformed voters.  
\begin{align*}
\text{If } p \geq \phi^*_{v1x0} &\text{ set } \phi_H = \phi^*_{v1x0}/p, \Rightarrow \phi = p \cdot \phi_H = \phi^*_{v1x0} \text{ so that } v^*_U = 0  \\
\text{If } p \geq \phi^*_{v1x1} &\text{ set } \phi_H = \phi^*_{v1x1}/p \Rightarrow \phi = p \cdot \phi_H = \phi^*_{v1x1} \text{ so that } v^*_U = 1, x^*_U = 1 
\end{align*}
\noindent If $p \geq \phi^*_{v1x0}$, $P_H$ prefers $\phi_H = \phi^*_{v1x0}/p$ over $\phi_H = 0$ if and only if:
\begin{align*}
EU_{P_H}[\phi_H = \phi^*_{v1x0}/p, v_U = 0] &> EU_{P_H}[\phi_H=0, v_U=1, x_U=0] = 2 \kappa_a \\ 
\pi (1-\kappa_a-\kappa_r)\left(\kappa_a - \frac{\phi^*_{v1x0}}{p}(\kappa_a + \kappa_r)\right) &> \frac{\phi^*_{v1x0}}{p} (\kappa_a + \kappa_r - 0.5)
\end{align*}
\noindent The above condition can be rewritten using the function $\Gamma$. The inequality holds if and only if $\Gamma$ is larger than zero:
\begin{align*}
0 <& \pi (1-\kappa_a-\kappa_r)\left(\kappa_a - \frac{\phi^*_{v1x0}}{p}(\kappa_a + \kappa_r)\right) -\frac{\phi^*_{v1x0}}{p}(\kappa_a + \kappa_r - 0.5)\\
0 <& \pi ( (1-\kappa_a-\kappa_r)(\pi\kappa_a(p(1+\kappa_a-\kappa_r)-(\kappa_a+\kappa_r))+p\kappa_ad-(\kappa_a+\kappa_r)(d-c)) - \kappa_a(\kappa_a+\kappa_r-0.5) ) \\ &- (\kappa_a+\kappa_r-0.5)(d-c) = \Gamma
\end{align*}
\noindent By assumption, $\kappa_a \in [0,1)$, $\kappa_r \in [0,1)$, $1-\kappa_a-\kappa_r > 0$, $\pi \in (0,1]$, $\kappa_a+\kappa_r \geq 0.5$, $d>c\geq=0$, and $p \in [0,1]$. Therefore, $\Gamma$ is decreasing in $d$ and increasing in $c$ and $p$. Also, the inequality $\Gamma>0$ holds only when $\Gamma$ is increasing in $\pi$. 

\par Replace $K = \kappa_a + \kappa_r \in [0.5,1)$ and $m = \kappa_a/(\kappa_a+\kappa_r) \in [0,1]$ in $\Gamma$. The previous inequality can be rewritten as follows:
\begin{align*}
0 <& \Gamma  \\
0 <& \pi ( (1-K)(\pi mK(p(1+mK-(1-m)K)-K)+pmKd-K(d-c)) - mK(K-0.5) ) \\ &- (K-0.5)(d-c) \\
0 <&\pi( K(1-K)(m(\pi(p-K(1+p(1-2m))) + pd) - (d-c)) - mK(K-0.5)) \\ &- (K-0.5)(d-c) = \Gamma_{1}\\
0 <&\pi( m(K(1-K)(\pi(p-K(1+p(1-2m))) + pd) - K(K-0.5)) - K(1-K)(d-c) ) \\ &- (K-0.5)(d-c) = \Gamma_{2}
\end{align*}
\noindent Since $(K-0.5)\geq0$, $1+p(1-2m)\geq 0$ by assumption, $\Gamma_{1}$ implies that $\Gamma$ is decreasing in $K$ and increasing in $K(1-K)$. Notice that $K(1-K)$ is maximized at $K=0.5$ and $K\geq0.5$ by assumption. This condition indicates that $\Gamma$ is weakly decreasing in $K$. Therefore, the inequality holds for sufficiently low $K$. 

\par Also, notice that $(K-0.5)(d-c)\geq0$, $K(1-K)(d-c)\geq 0$, and $m\geq0$ in $\Gamma_2$. This quality implies that $K(1-K)(\pi(p-K(1+p(1-2m))) + pd) - K(K-0.5)>0$ must hold for the inequality to be satisfied. This condition implies that $\Gamma$ is increasing in $m$ whenever the inequality is satisfied. Therefore, the inequality holds for sufficiently high $m$.

\par In sum, when $p \geq \phi^*_{v1x0}$ and $\kappa_a+\kappa_r \geq 0.5$, $P_H$ prefers $\phi_H=\phi^*_{v1x0}/p$ over $\phi_H=0$ for sufficiently high $p$, $\pi$, $m = \kappa_a/(\kappa_a+\kappa_r)$, and $c$ and sufficiently low $K = \kappa_a + \kappa_r$ and $d$. 

\par If $p \geq \phi^*_{v1x1}$, $P_H$ prefers $\phi_H = \phi^*_{v1x1}/p$ over $\phi_H = 0$ if and only if:
\begin{align*}
EU_{P_H}[\phi_H = \phi^*_{v1x1}/p, v_U = 1, x_U = 1] &> EU_{P_H}[\phi_H=0, v_U=1, x_U=0] = 2 \kappa_a \\ 
\pi \left(1-\frac{\phi^*_{v1x1}}{p}\right)(1-\kappa_a-\kappa_r) &> \frac{\phi^*_{v1x1}}{p} (\kappa_a + \kappa_r - 0.5) \\
\pi (1-\kappa_a-\kappa_r) &> \frac{\phi^*_{v1x1}}{p} (\pi(1-\kappa_a-\kappa_r) + (\kappa_a + \kappa_r - 0.5)) \\
\end{align*}
\noindent The above inequality can be rewritten by using the function $\Theta$. The inequality holds if and only if $\Theta$ is larger than zero:
\begin{align*}
0 <& \pi (1-\kappa_a-\kappa_r) - \frac{\phi^*_{v1x1}}{p} (\pi(1-\kappa_a-\kappa_r) + (\kappa_a + \kappa_r - 0.5)) \\
0 <& \pi( (1-\kappa_a-\kappa_r)(\pi (p\kappa_r-(1-p)(1-\kappa_a)) + pd - c) - (1-\kappa_a)(\kappa_a+\kappa_r-0.5)) \\ &- (\kappa_a+\kappa_r-0.5)c = \Theta 
\end{align*}
\noindent By assumption, $p\in[0,1]$, $1-\kappa_a-\kappa_r > 0$, $\pi \in (0,1]$, and $\kappa_a+\kappa_r \geq 0.5$. Therefore, $\Theta$ is increasing in $p$ and $d$ and decreasing in $c$. Also, the inequality $\Theta>0$ holds only when $\Theta$ is increasing in $\pi$ at $\pi=0$. 

\par Replace $K = \kappa_a + \kappa_r \in [0.5,1)$ and $m = \kappa_a/(\kappa_a+\kappa_r) \in [0,1]$ in $\Theta$. The previous inequality can be rewritten as follows:
\begin{align*}
0 <&\Theta \\
0 <& \pi( (1-K)(\pi (p(1-m)K-(1-p)(1-mK)) + pd - c) - (1-mK)(K-0.5)) - (K-0.5)c\\ 
%0 <& (1-K)(\pi (p(1-m)K-(1-p)(1-mK)) + pd - c) - \frac{K(\pi(1-m(K-0.5))+c)-0.5(c-\pi)}{\pi} \\
0 <& K(1-K)\pi^2 (p(1-m)+m(1-p)) - K(\pi(1-m(K-0.5)-\pi(1-p)+pd)+c(1-\pi)) \\
& - \pi(1.5-p(1+d)+c)+0.5c 
\end{align*}
\noindent Since $\pi \in (0,1]$, $p(1-m)+m(1-p)>0$, and $K\in[0.5,1)$, the function $K(1-K)\pi^2 (p(1-m)+m(1-p))$ is decreasing in $K$ (maximized at $0.5$). Also, since $m \in [0,1]$, $p>0.5=\phi^*_x$, $\pi \in (0,1]$, $d>0$, and $c\geq0$, $-K(\pi(1-m(K-0.5)-\pi(1-p)+pd)+c(1-\pi))$ is decreasing in $K$. All conditions imply that the function $\Theta$ is decreasing in $K$. Therefore, the inequality holds for sufficiently low $K \geq 0.5$.

\par Also, the following function extracts the part of $\Theta$ relevant to $m$. 
\begin{align*}
\Lambda = m \pi K( \pi(1-K)(1-2p)+(K-0.5))
\end{align*}
\noindent $\Lambda$ is increasing in $m$ if and only if: 
\begin{align*}
\pi(1-K)(1-2p)+(K-0.5) &> 0 \\
K &>\frac{0.5 + \pi(2p-1)}{1 + \pi(2p-1)} \tag{Note:$2p-1\geq0$}
\end{align*}
\noindent Therefore, if $K >\frac{0.5 + \pi(2p-1)}{1 + \pi(2p-1)}$, the inequality $\Theta > 0$ holds for sufficiently high $m$. If $K \leq \frac{0.5 + \pi(2p-1)}{1 + \pi(2p-1)}$, the inequality holds for sufficiently low $m$. 

\par In sum, when $p \geq \phi^*_{v1x1}$ and $\kappa_a+\kappa_r \geq 0.5$, $P_H$ prefers $\phi_H=\phi^*_{v1x1}/p$ over $\phi_H=0$ for sufficiently high $p$, $\pi$, and $d$ and sufficiently low $K = \kappa_a + \kappa_r$ and $c$. If $K >\frac{0.5 + \pi(2p-1)}{1 + \pi(2p-1)}$, this condition holds for sufficiently high $m = \kappa_a/(\kappa_a+\kappa_r)$. If $K \leq \frac{0.5 + \pi(2p-1)}{1 + \pi(2p-1)}$, this condition holds for sufficiently low $m = \kappa_a/(\kappa_a+\kappa_r)$.

\par If the abstention interval does not exist for non-ideologue uninformed voters (i.e., $\phi^*_{v1x0}=\phi^*_{v1x1}=\phi^*_x=0.5$) and $p \geq 0.5$, $P_H$ prefers $\phi_H = \phi^*_{v1x1}/p$ over $\phi_H = 0$ if and only if:
\begin{align*}
EU_{P_H}[\phi_H = 1/2p, v_U = 1, x_U = 1] &> EU_{P_H}[\phi_H=0, v_U=1, x_U=0] = 2 \kappa_a \\ 
%\pi \left(1-\frac{1}{2p}\right)(1-\kappa_a-\kappa_r) &> \frac{1}{2p} (\kappa_a + \kappa_r - 0.5) \\
\pi (1-\kappa_a-\kappa_r) &> \frac{1}{2p} (\pi(1-\kappa_a-\kappa_r) + (\kappa_a + \kappa_r - 0.5)) \\
p &> \frac{1}{2}+ \frac{\kappa_a + \kappa_r - 0.5}{2\pi(1-\kappa_a-\kappa_r)}
\end{align*}
\noindent The above function implies that the inequality holds for sufficiently high $p$ and $\pi$ and sufficiently low $K = \kappa_a + \kappa_r$.

% \par Lastly, suppose that both $EU_{P_H}[\phi_H = \phi^*_{v1x0}/p, v_U = 0]$ and  $EU_{P_H}[\phi_H=\phi^*_{v1x1}, v_U=1, x_U=1]$ are larger than $EU_{P_H}[\phi_H = 0, v_U = 1, x_U=0]$.Then, $P_H$ prefers $\phi_H=\phi^*_{v1x1}$ over $\phi_H=\phi^*_{v1x0}$ if and only if:   
% \begin{align*}
% EU_{P_H}[\phi_H = \phi^*_{v1x0}/p, v_U = 0] &> EU_{P_H}[\phi_H=0, v_U=1, x_U=0]
% \end{align*}

\hfill $\blacksquare$

\subsubsection{The Proof of Proposition 6}

\par It directly follows from the proof of Proposition 5 that:
\begin{itemize}
	\item $EU_{P_H}[\phi_H = \phi^*_{v1x0}/p, v_U = 0] > EU_{P_H}[\phi_H=0, v_U=1, x_U=0]=2\kappa_a$ occurs under sufficiently low $d$ and sufficiently high $c$. 
	\item $EU_{P_H}[\phi_H = \phi^*_{v1x0}/p, v_U = 1, x_U = 1] > EU_{P_H}[\phi_H=0, v_U=1, x_U=0]=2\kappa_a$ occurs under sufficiently high $d$ and sufficiently low $c$.
\end{itemize}
\noindent The above two facts imply that $EU_{P_H}[\phi_H = \phi^*_{v1x0}/p, v_U = 1, x_U = 1] > EU_{P_H}[\phi_H=0, v_U=0]$ occurs under sufficiently high $d$ and sufficiently low $c$.

\par Also, it is shown in Proposition 1 that:
\begin{itemize}
	\item The discouraged abstention (i.e., $\phi^*_{v1x1}$ is decreasing in $\pi$ and $\phi^*_{v1x1}$ is increasing in $\pi$) is available under sufficiently low $d$ and/or sufficiently high $c$. 
	\item The delegatory abstention (i.e., $\phi^*_{v1x1}$ is increasing in $\pi$ and $\phi^*_{v1x1}$ is decreasing in $\pi$) is available under sufficiently high $d$ and/or sufficiently low $c$.
\end{itemize}

\par Summarizing above facts, holding other parameters constant, sufficiently low $d$ and sufficiently high $c$ make both $\phi^*_H = \phi^*_{v1x0}$ and discouraged abstention available; sufficiently high $d$ and sufficiently low $c$ make both $\phi^*_H = \phi^*_{v1x1}$ and delegatory abstention available. 

\hfill $\blacksquare$

\subsubsection{The Proof of Proposition 7}

\par Suppose that $\kappa_a + \kappa_r \geq 0.5$ and abstention is not allowed. If $\pi=0$, $P_H$ chooses a low-quality policy for sure (i.e., $\phi^*_H=0 \Rightarrow \phi^* = p \cdot \phi^*_H = 0$). Non-ideologue uninformed voters choose rejection (i.e., $v_U=1$, $x_U=0$) for sure (Lemma 3). Under this condition, the expected utility of $P_H$ can be calculated as follows:
\begin{align*}
EU_{P_H}[\phi_H=0, v_U=1, x_U=0] = 2 \kappa_a
\end{align*}
\noindent If $\pi>0$, It may be possible for $P_H$ to influence the decision of non-ideologue uninformed voters by setting $\phi_H>0$. Following statements describe when and how $P_H$ can influence the decision of non-ideologue uninformed voters. Since the expected cost of policymaking is increasing in $\phi_H$, $P_H$ chooses the lowest possible value of $\phi_H$ that can ensure the particular behavior of uninformed voters.  
\begin{align*}
\text{If } p \geq \phi^*_{x} &\text{ set } \phi_H = \phi^*_{x}/p, \Rightarrow \phi = p \cdot \phi_H = \phi^*_{x} \text{ so that } v^*_U = 1, x^*_U = 1 
\end{align*}
\noindent If $p \geq \phi^*_{x}$, $P_H$ prefers $\phi_H = \phi^*_{x}/p$ over $\phi_H = 0$ if and only if:
\begin{align*}
EU_{P_H}[\phi_H = \phi^*_{x}/p, v_U = 0] &> EU_{P_H}[\phi_H=0, v_U=1, x_U=0] = 2 \kappa_a \\ 
\pi (1-\kappa_a-\kappa_r)\left(\kappa_a - \frac{\phi^*_{x}}{p}(\kappa_a + \kappa_r)\right) &> \frac{\phi^*_{x}}{p} (\kappa_a + \kappa_r - 0.5) \\
\pi (1-\kappa_a-\kappa_r)\left(\kappa_a - \frac{1}{2p}(\kappa_a + \kappa_r)\right) &> \frac{1}{2p} (\kappa_a + \kappa_r - 0.5)
\end{align*}

\hfill $\blacksquare$

\subsubsection{The Proof of Proposition 8}

\par Suppose that, in the extended accountability game, uninformed voters $U$ sets $\pi$ endogenously at the start of the game (before observing $p$ and $\beta_U$). Also assume that $\kappa_a + \kappa_r \geq 0.5$, so that $\phi^*_H=0$ when $\pi=0$. 

\par To start with, the expected welfare of uninformed voters $U$ under $\phi^*_H=0=\phi^*$ and thus $v^*_U=1$ and $x^*_U=0$ can be calculated as follows:

\begin{align*}
	Eu_U[\phi^*_H=0] =& \kappa_a \cdot \kappa_a [ A - 1 + d - c] + \\
	&\kappa_a \cdot \kappa_r [ \pi(A-1) + d - c] + \\
	&\kappa_a \cdot (1-\kappa_a-\kappa_r) [ \pi(A-1) + d - c] + \\
	&\kappa_r \cdot \kappa_a [ (1-\pi)(R-1) + d - c] + \\
	&\kappa_r \cdot \kappa_r [ d - c ] + \\
	&\kappa_r \cdot (1-\kappa_a-\kappa_r) [ d - c ] + \\
	&(1-\kappa_a-\kappa_r) \cdot \kappa_a [ (1-\pi)(-1) + d - c] + \\
	&(1-\kappa_a-\kappa_r) \cdot \kappa_r [ d - c ] + \\
	&(1-\kappa_a-\kappa_r) \cdot (1-\kappa_a-\kappa_r) [ d - c ] \\
	=& d - c + \kappa_a^2 (A - 1) + \\
	&\pi \kappa_a (1-\kappa_a) (A-1) + \\ 
	&(1-\pi)\kappa_a (\kappa_r(R-1) - (1-\kappa_a-\kappa_r))
\end{align*}

\noindent From the above equation, given that $d \geq c\geq0$, $0<\kappa_a<1$, $0<\kappa_r<1$, $\kappa_a+\kappa_r<1$, $A>1$, and $R<-1$, $Eu_U[\phi^*_H=0]$ is increasing in $\pi$. 

\par If, under delegatory context, $\phi^*_H = \phi^*_{v1x1}/p \Rightarrow \phi^* = \phi^*_{v1x1}$ and thus $v^*_U=1$ and $x^*_U=1$, the expected welfare of $U$ can be calculated as follows:

\begin{align*}
	Eu_U[\phi^*_H = \phi^*_{v1x1}/p] =& \kappa_a \cdot \kappa_a [ A + 2\phi^*_{v1x1} - 1 + d - c] + \\
	&\kappa_a \cdot \kappa_r [ \pi(A+2\phi^*_{v1x1}-1) + d - c] + \\
	&\kappa_a \cdot (1-\kappa_a-\kappa_r) [ \pi(A+2\phi^*_{v1x1}-1) + (1-\pi)\phi^*_{v1x1}(A+1) + d - c] + \\
	&\kappa_r \cdot \kappa_a [ (1-\pi)(R+2\phi^*_{v1x1}-1) + d - c] + \\
	&\kappa_r \cdot \kappa_r [ d - c ] + \\
	&\kappa_r \cdot (1-\kappa_a-\kappa_r) [ (1-\pi)\phi^*_{v1x1}(R+1) + d - c] + \\
	&(1-\kappa_a-\kappa_r) \cdot \kappa_a [ 2\phi^*_{v1x1}-1 + \phi^*_{v1x1} d - c] + \\
	&(1-\kappa_a-\kappa_r) \cdot \kappa_r [ \pi (2\phi^*_{v1x1}-1) + \phi^*_{v1x1} d - c] + \\
	&(1-\kappa_a-\kappa_r) \cdot (1-\kappa_a-\kappa_r) [ \pi (2\phi^*_{v1x1}-1)+ (1-\pi)\phi^*_{v1x1} + \phi^*_{v1x1} d - c ] \\
	=& (\kappa_r+\kappa_a)(d-c) + \\
	&(1-\kappa_a-\kappa_r)\underline{\pi\kappa_a(A+2\phi^*_{v1x1}-1)}+\\
	&(1-\kappa_a-\kappa_r)\underline{(1-\pi)\phi^*_{v1x1}(\kappa_a(1+A)+\kappa_r(1+R))} + \\
	&\kappa_a^2(A+2\phi^*_{v1x1}-1) + \\
	&(1-\kappa_a-\kappa_r)(\phi^*_{v1x1}(1-\kappa_r+\kappa_a)-\kappa_a)
\end{align*}

\noindent Under delegatory context, $\phi^*_{v1x1}$ is increasing in $\pi$. Then, focus on the underlined parts in the above equation. If the below inequality holds, the entire function of $Eu_U[\phi^*_H = \phi^*_{v1x1}/p]$ is increasing in $\pi$: 

\begin{align*}
	\kappa_a(A+2\phi^*_{v1x1}-1) &> \phi^*_{v1x1}(\kappa_a(1+A)+\kappa_r(1+R))\\
	(1-\phi^*_{v1x1})\kappa_a(A-1) &> \kappa_r(1+R)
\end{align*}

\noindent The above inequality always holds because $1+R<0$, $A-1>0$, and $1-\phi^*_{v1x1}>0$ by assumption. 

\par If, under discouraged context, $\phi^*_H = \phi^*_{v1x0}/p \Rightarrow \phi^* = \phi^*_{v1x0}$ and thus $v^*_U=0$, the expected welfare of $U$ can be calculated as follows:

\begin{align*}
	Eu_U[\phi^*_H = \phi^*_{v1x0}/p] =& \kappa_a \cdot \kappa_a [ A + 2\phi^*_{v1x0} - 1 + d - c] + \\
	&\kappa_a \cdot \kappa_r [ \pi(A+2\phi^*_{v1x0}-1) + d - c] + \\
	&\kappa_a \cdot (1-\kappa_a-\kappa_r) [ \pi(A+2\phi^*_{v1x0}-1) + (1-\pi)\phi^*_{v1x0}(A+1) + d - c] + \\
	&\kappa_r \cdot \kappa_a [ (1-\pi)(R+2\phi^*_{v1x0}-1) + d - c] + \\
	&\kappa_r \cdot \kappa_r [ d - c ] + \\
	&\kappa_r \cdot (1-\kappa_a-\kappa_r) [ (1-\pi)\phi^*_{v1x0}(R+1) + d - c] + \\
	&(1-\kappa_a-\kappa_r) \cdot \kappa_a [ 2\phi^*_{v1x0}-1 ] + \\
	&(1-\kappa_a-\kappa_r) \cdot \kappa_r [ 0 ] + \\
	&(1-\kappa_a-\kappa_r) \cdot (1-\kappa_a-\kappa_r) [ \phi^*_{v1x0} ] \\
	=& (\kappa_r+\kappa_a)(d-c) + \\
	&\kappa_a(A+2\phi^*_{v1x0}-1)(\kappa_a+\pi\kappa_r) + \\
	&\kappa_a(1-\kappa_a-\kappa_r)(\pi(A+2\phi^*_{v1x0}-1) + (1-\pi)\phi^*_{v1x0}(A+1)) + \\
	&\kappa_r(1-\pi)(\kappa_a(R+2\phi^*_{v1x0}-1)+(1-\kappa_a-\kappa_r)\phi^*_{v1x0}(R+1)) + \\
	&(1-\kappa_a-\kappa_r)(\kappa_a(2\phi^*_{v1x0}-1)+(1-\kappa_a-\kappa_r)\phi^*_{v1x0}) 
\end{align*}

\noindent Under discouraged context, $\phi^*_{v1x0}$ is increasing in $\pi$. Then, since $A+2\phi^*_{v1x0}-1>\phi^*_{v1x0}(A+1)>0$, $\kappa_a(R+2\phi^*_{v1x0}-1)<0$, $(1-\kappa_a-\kappa_r)\phi^*_{v1x0}(R+1)<0$, the entire function of $Eu_U[\phi^*_H = \phi^*_{v1x0}/p]$ is increasing in $\pi$. 

\par From the above, $Eu_U[\phi^*_H=0]$, $Eu_U[\phi^*_H = \phi^*_{v1x1}/p]$ under delegatory context, and $Eu_U[\phi^*_H = \phi^*_{v1x0}/p]$ under discouraged context are all increasing in $\pi$. Therefore, following statement stands if accountability improvement is available at higher range of $\pi \in (\underline{\pi},1]$.

\begin{align*}
	\pi* &= 
	\begin{cases}
		1 \text{ if } &Eu_U[\phi^*_H = \phi^*_{v1x1}/p, \pi=1] \geq Eu_U[\phi^*_H=0,\pi=\underline{\pi}] \text{ or }\\
		&Eu_U[\phi^*_H = \phi^*_{v1x0}/p, \pi=1] \geq Eu_U[\phi^*_H=0,\pi=\underline{\pi}]\\
		\underline{\pi} \text{ Otherwise} &\text{ }\\
	\end{cases}
\end{align*}

\par Then, $Eu_U[\phi^*_H = \phi^*_{v1x1}/p, \pi=1] \geq Eu_U[\phi^*_H=0,\pi=\underline{\pi}]$ iff: 

\begin{align*}
	&Eu_U[\phi^*_H = \phi^*_{v1x1}/p, \pi=1] - Eu_U[\phi^*_H=0,\pi=\underline{\pi}] \geq 0 \\
	&(1-\kappa_r)(2\phi^*_{v1x1}[\pi=1]-1)\\
	&+\kappa_a(\kappa_a + (1-\kappa_a)(\underline{\pi} + (1-\underline{\pi})A)) \\
	&+(1-\underline{\pi}) \kappa_a (-\kappa_r(R-2)-(1-\kappa_a))\\
	&-(1-\kappa_a-\kappa_r)(1-\phi^*_{v1x1}[\pi=1])d \geq 0
\end{align*}

\par It directly follows from the above function that the difference is increasing in $A$ and decreasing in $R$. In addition, following statements hold:

\begin{enumerate}
	\item Given that $R<-1$ and $A>1$, the above function is increasing or decreasing in $\underline{\pi}$ (decreasing if $R < 2 - (1-\kappa_a)/\kappa_r$, may be increasing otherwise). If the improvement occurs at $\phi^*_{v1x1}/p$ when $\pi \in (\underline{\pi},1]$), $\underline{\pi}$ is the largest value of $\pi$ that satisfies $\Theta\leq0$ (see Proposition 5 for the definition of $\Theta$). Then, the larger the $d$, the lower the maximum value of $\pi$ that satisfies $\Theta\leq0$ (i.e., $\underline{\pi}$ is decreasing in $d$) and minimized at $1/2$ (when abstention interval is not available, from Lemma 3). 
	\item The above function is increasing in $\phi^*_{v1x1}[\pi=1]$. $\phi^*_{v1x1}[\pi=1]$ is decreasing in $d$ and minimized at $1/2$ (when abstention interval is not available, from Lemma 3). 
	\item $-(1-\kappa_a-\kappa_r)(1-\phi^*_{v1x1}[\pi=1])d$ is infinitely decreasing in $d$, with no minimum value. 
\end{enumerate}

\noindent From the above three statements, holding other parameters, $Eu_U[\phi^*_H = \phi^*_{v1x1}/p, \pi=1] - Eu_U[\phi^*_H=0,\pi=\underline{\pi}]$ may be increasing or decreasing in $d$ for the lower range of $d$ (Statement 1 and 2) but decreasing in $d$ for sure when $d$ is overly high (Statement 3). 
 
\par Also, $Eu_U[\phi^*_H = \phi^*_{v1x0}/p, \pi=1] \geq Eu_U[\phi^*_H=0,\pi=\underline{\pi}]$ under discouraged context iff: 

\begin{align*}
	&Eu_U[\phi^*_H = \phi^*_{v1x1}/p, \pi=1] - Eu_U[\phi^*_H=0,\pi=\underline{\pi}] \geq 0 \\
	&\kappa_a(1-\kappa_a-\kappa_r)(2\phi^*_{v1x0}[\pi=1]-1)\\
	&+\kappa_a(\kappa_a + (1-\kappa_a)(\underline{\pi} + (1-\underline{\pi})A)) \\
	&+(1-\underline{\pi}) \kappa_a (-\kappa_r(R-2)-(1-\kappa_a))\\
	&+(1-\kappa_a-\kappa_r)((1-\kappa_a-\kappa_r)\phi^*_{v1x0}[\pi=1]-(d-c))
\end{align*}

\par It directly follows from the above function that the difference is increasing in $A$ and decreasing in $R$. In addition, following statements hold:

\begin{enumerate}
	\item Given that $R<-1$ and $A>1$, the above function is increasing or decreasing in $\underline{\pi}$ (decreasing if $R < 2 - (1-\kappa_a)/\kappa_r$, may be increasing otherwise). If the improvement occurs at $\phi^*_{v1x0}/p$ when $\pi \in (\underline{\pi},1]$), $\underline{\pi}$ is the largest value of $\pi$ that satisfies $\Gamma\leq0$ (see Proposition 5 for the definition of $\Gamma$). Then, the larger the $d$, the higher the maximum value of $\pi$ that satisfies $\Gamma\leq0$ (i.e., $\underline{\pi}$ is increasing in $d$), but $\underline{\pi}<1/2$ must be supported (otherwise the improvement to $\phi^*_{v1x0}/p$ is not available, Lemma 3). 
	\item The above function is increasing in $\phi^*_{v1x0}[\pi=1]$. $\phi^*_{v1x0}[\pi=1]$ is increasing in $d$ but must stay $\underline{\pi}<1/2$ (otherwise the improvement to $\phi^*_{v1x0}/p$ is not available, Lemma 3). 
	\item $(1-\kappa_a-\kappa_r)((1-\kappa_a-\kappa_r)\phi^*_{v1x0}[\pi=1]-(d-c))$ is infinitely decreasing in $d$ (as long as $d/c$ ratio falls within the range of discouraged context), with no minimum value.
\end{enumerate}

\noindent From the above three statements, holding other parameters, $Eu_U[\phi^*_H = \phi^*_{v1x0}/p, \pi=1] - Eu_U[\phi^*_H=0,\pi=\underline{\pi}]$ may be increasing or decreasing in $d$ for the lower range of $d$ (Statement 1 and 2) but decreasing in $d$ for sure for overly high $d$ (Statement 3).

\par In summary, it is shown that, under the following assumptions:

\begin{itemize}
	\item Uninformed voters $U$ sets $\pi$ endogenously at the start of the extended accountability game (before observing $p$ and $\beta_U$). 
	\item $\kappa_a + \kappa_r \geq 0.5$, so that $P_H$ sets $\phi^*_H=0$ when $\pi=0$. 
	\item Under delegatory context, $P_H$ deviates to $\phi^*_{v1x1}/p$ for $\pi \in (\underline{\pi},1]$. 
	\item Under discouraged context, $P_H$ deviates to $\phi^*_{v1x0}/p$ for $\pi \in (\underline{\pi},1]$.
\end{itemize}

\noindent Uninformed voters $U$ sets $\pi^*=1$ unless $A$ is too high, $R$ is too low, and/or $d$ is overly high, and sets $\pi^*=\underline{\pi}$ otherwise.

\hfill $\blacksquare$

% \subsubsection{The Proof of Proposition 8}

% \par To start with, the expected welfare of voters $I$ and $U$ under $\phi^*_H=0=\phi^*$ and thus $v^*_U=1$ and $x^*_U=0$ can be calculated as follows:

% \begin{align*}
% 	Eu_I[\phi^*_H=0] =& \kappa_a \cdot \kappa_a [ A - 1 + d - c] + \\
% 	&\kappa_a \cdot \kappa_r [ \pi(R-1) + d - c] + \\
% 	&\kappa_a \cdot (1-\kappa_a-\kappa_r) [ \pi(-1) + d - c] + \\
% 	&\kappa_r \cdot \kappa_a [ (1-\pi)(A-1) + d - c] + \\
% 	&\kappa_r \cdot \kappa_r [ d - c ] + \\
% 	&\kappa_r \cdot (1-\kappa_a-\kappa_r) [ d - c ] + \\
% 	&(1-\kappa_a-\kappa_r) \cdot \kappa_a [ (1-\pi)(A-1) + d - c] + \\
% 	&(1-\kappa_a-\kappa_r) \cdot \kappa_r [ d - c ] + \\
% 	&(1-\kappa_a-\kappa_r) \cdot (1-\kappa_a-\kappa_r) [ d - c ] 
% \end{align*}

% \begin{align*}
% 	Eu_U[\phi^*_H=0] =& \kappa_a \cdot \kappa_a [ A - 1 + d - c] + \\
% 	&\kappa_a \cdot \kappa_r [ \pi(A-1) + d - c] + \\
% 	&\kappa_a \cdot (1-\kappa_a-\kappa_r) [ \pi(A-1) + d - c] + \\
% 	&\kappa_r \cdot \kappa_a [ (1-\pi)(R-1) + d - c] + \\
% 	&\kappa_r \cdot \kappa_r [ d - c ] + \\
% 	&\kappa_r \cdot (1-\kappa_a-\kappa_r) [ d - c ] + \\
% 	&(1-\kappa_a-\kappa_r) \cdot \kappa_a [ (1-\pi)(-1) + d - c] + \\
% 	&(1-\kappa_a-\kappa_r) \cdot \kappa_r [ d - c ] + \\
% 	&(1-\kappa_a-\kappa_r) \cdot (1-\kappa_a-\kappa_r) [ d - c ] 
% \end{align*}

% %\noindent If $\phi^*_H = \phi^*_{v1x0}/p \Rightarrow \phi^* = \phi^*_{v1x0}$ and thus $v^*_U=0$:
% %\begin{align*}
% % 	Eu_U[\phi^*_H = \phi^*_{v1x0}/p] =& \kappa_a \cdot \kappa_a [ A + 2\phi^*_{v1x0} - 1 + d - c] + \\
% % 	&\kappa_a \cdot \kappa_r [ \pi(A+2\phi^*_{v1x0}-1) + d - c] + \\
% % 	&\kappa_a \cdot (1-\kappa_a-\kappa_r) [ \pi(A+2\phi^*_{v1x0}-1) + (1-\pi)\phi^*_{v1x0}(A+1) + d - c] + \\
% % 	&\kappa_r \cdot \kappa_a [ (1-\pi)(R+2\phi^*_{v1x0}-1) + d - c] + \\
% % 	&\kappa_r \cdot \kappa_r [ d - c ] + \\
% % 	&\kappa_r \cdot (1-\kappa_a-\kappa_r) [ (1-\pi)\phi^*_{v1x0}(R+1) + d - c] + \\
% % 	&(1-\kappa_a-\kappa_r) \cdot \kappa_a [ 2\phi^*_{v1x0}-1 ] + \\
% % 	&(1-\kappa_a-\kappa_r) \cdot \kappa_r [ 0 ] + \\
% % 	&(1-\kappa_a-\kappa_r) \cdot (1-\kappa_a-\kappa_r) [ \phi^*_{v1x0} ] 
% % \end{align*}

% \noindent Then, if $\phi^*_H = \phi^*_{v1x1}/p \Rightarrow \phi^* = \phi^*_{v1x1}$ and thus $v^*_U=1$ and $x^*_U=1$:

% \begin{align*}
% 	Eu_I[\phi^*_H = \phi^*_{v1x1}/p] =& \kappa_a \cdot \kappa_a [ A + 2\phi^*_{v1x1} - 1 + d - c] + \\
% 	&\kappa_a \cdot \kappa_r [ \pi(R+2\phi^*_{v1x1}-1) + d - c] + \\
% 	&\kappa_a \cdot (1-\kappa_a-\kappa_r) [ \pi(2\phi^*_{v1x1}-1) + (1-\pi)\phi^*_{v1x1} + d - c] + \\
% 	&\kappa_r \cdot \kappa_a [ (1-\pi)(A+2\phi^*_{v1x1}-1) + d - c] + \\
% 	&\kappa_r \cdot \kappa_r [ d - c ] + \\
% 	&\kappa_r \cdot (1-\kappa_a-\kappa_r) [ (1-\pi)\phi^*_{v1x1} + d - c] + \\
% 	&(1-\kappa_a-\kappa_r) \cdot \kappa_a [ A + 2\phi^*_{v1x1}-1 + d - c] + \\
% 	&(1-\kappa_a-\kappa_r) \cdot \kappa_r [ \pi (R+2\phi^*_{v1x1}-1) + d - c] + \\
% 	&(1-\kappa_a-\kappa_r) \cdot (1-\kappa_a-\kappa_r) [ \pi (2\phi^*_{v1x1}-1)+ (1-\pi)\phi^*_{v1x1} + d - c ] 
% \end{align*}

% \begin{align*}
% 	Eu_U[\phi^*_H = \phi^*_{v1x1}/p] =& \kappa_a \cdot \kappa_a [ A + 2\phi^*_{v1x1} - 1 + d - c] + \\
% 	&\kappa_a \cdot \kappa_r [ \pi(A+2\phi^*_{v1x1}-1) + d - c] + \\
% 	&\kappa_a \cdot (1-\kappa_a-\kappa_r) [ \pi(A+2\phi^*_{v1x1}-1) + (1-\pi)\phi^*_{v1x1}(A+1) + d - c] + \\
% 	&\kappa_r \cdot \kappa_a [ (1-\pi)(R+2\phi^*_{v1x1}-1) + d - c] + \\
% 	&\kappa_r \cdot \kappa_r [ d - c ] + \\
% 	&\kappa_r \cdot (1-\kappa_a-\kappa_r) [ (1-\pi)\phi^*_{v1x1}(R+1) + d - c] + \\
% 	&(1-\kappa_a-\kappa_r) \cdot \kappa_a [ 2\phi^*_{v1x1}-1 + \phi^*_{v1x1} d - c] + \\
% 	&(1-\kappa_a-\kappa_r) \cdot \kappa_r [ \pi (2\phi^*_{v1x1}-1) + \phi^*_{v1x1} d - c] + \\
% 	&(1-\kappa_a-\kappa_r) \cdot (1-\kappa_a-\kappa_r) [ \pi (2\phi^*_{v1x1}-1)+ (1-\pi)\phi^*_{v1x1} + \phi^*_{v1x1} d - c ] 
% \end{align*}

% \par From the above, the change in voter welfare in response to accountability improvement from from $\phi^*_H=0$ to $\phi^*_H=\phi^*_{v1x1}$ can be calculated as follows:

% \begin{align*}
% 	Eu_I[\phi^*_H = \phi^*_{v1x1}/p] &- Eu_I[\phi^*_H = 0]\\
% 	=& \kappa_a \cdot \kappa_a [ 2\phi^*_{v1x1}] + \\
% 	&\kappa_a \cdot \kappa_r [ 2 \pi \phi^*_{v1x1} ] + \\
% 	&\kappa_a \cdot (1-\kappa_a-\kappa_r) [ 2 \pi \phi^*_{v1x1} + (1-\pi)\phi^*_{v1x1}] + \\
% 	&\kappa_r \cdot \kappa_a [ 2(1-\pi)\phi^*_{v1x1} ] + \\
% 	&\kappa_r \cdot \kappa_r [ 0 ] + \\
% 	&\kappa_r \cdot (1-\kappa_a-\kappa_r) [ (1-\pi)\phi^*_{v1x1}] + \\
% 	&(1-\kappa_a-\kappa_r) \cdot \kappa_a [ \pi (A-1) + 2\phi^*_{v1x1} ] + \\
% 	&(1-\kappa_a-\kappa_r) \cdot \kappa_r [ \pi (R+2\phi^*_{v1x1}-1)  ] + \\
% 	&(1-\kappa_a-\kappa_r) \cdot (1-\kappa_a-\kappa_r) [ \pi (2\phi^*_{v1x1}-1) + (1-\pi)\phi^*_{v1x1} ] 
% \end{align*}

% \noindent It diretly follows from the above equation that $Eu_I[\phi^*_H = \phi^*_{v1x1}/p] - Eu_I[\phi^*_H = 0]$ is increasing in $\phi^*_{v1x1}$. Since $\phi^*_{v1x1}$ under $\beta_U=0$ is decreasing in $d$ (Lemma 3), $Eu_I[\phi^*_H = \phi^*_{v1x1}/p] - Eu_I[\phi^*_H = 0]$ is decreasing in $d$.

% \par For uninformed voters:

% % \begin{align*}
% % 	Eu_U[\phi^*_H = \phi^*_{v1x1}/p] &- Eu_U[\phi^*_H = 0] \\
% % 	=& \kappa_a \cdot \kappa_a [ 2\phi^*_{v1x1}] + \\
% % 	&\kappa_a \cdot \kappa_r [ 2 \pi \phi^*_{v1x1} ] + \\
% % 	&\kappa_a \cdot (1-\kappa_a-\kappa_r) [ 2 \pi \phi^*_{v1x1} + (1-\pi)\phi^*_{v1x1}(A+1)] + \\
% % 	&\kappa_r \cdot \kappa_a [ 2(1-\pi)\phi^*_{v1x1} ] + \\
% % 	&\kappa_r \cdot \kappa_r [ 0 ] + \\
% % 	&\kappa_r \cdot (1-\kappa_a-\kappa_r) [ (1-\pi)\phi^*_{v1x1}(R+1)] + \\
% % 	&(1-\kappa_a-\kappa_r) \cdot \kappa_a [ \pi (-1) + 2\phi^*_{v1x1} - (1-\phi^*_{v1x1})d ] + \\
% % 	&(1-\kappa_a-\kappa_r) \cdot \kappa_r [ \pi (2\phi^*_{v1x1}-1) - (1-\phi^*_{v1x1})d ] + \\
% % 	&(1-\kappa_a-\kappa_r) \cdot (1-\kappa_a-\kappa_r) [ \pi (2\phi^*_{v1x1}-1) + (1-\pi)\phi^*_{v1x1} - (1-\phi^*_{v1x1})d ] \\
% % 	=& \kappa_a \cdot \kappa_a [ 2\phi^*_{v1x1}] + \kappa_a \cdot \kappa_r [ 2 \pi \phi^*_{v1x1} ] + \\
% % 	&\kappa_a \cdot (1-\kappa_a-\kappa_r) [ 2 \pi \phi^*_{v1x1} + (1-\pi)\phi^*_{v1x1}(A+1)] + \\
% % 	&\kappa_r \cdot \kappa_a [ 2(1-\pi)\phi^*_{v1x1} ] + \\
% % 	&(1-\kappa_a-\kappa_r) \cdot \kappa_a [ \pi (-1) + 2\phi^*_{v1x1} - (1-\phi^*_{v1x1})d ] + \\
% % 	&(1-\kappa_a-\kappa_r) \cdot (1-\kappa_a-\kappa_r) [ \pi (2\phi^*_{v1x1}-1) + (1-\pi)\phi^*_{v1x1} - (1-\phi^*_{v1x1})d ] + \\
% % 	&(1-\kappa_a-\kappa_r) \cdot \kappa_r [ \{ \underline{(1-\pi)R + 1 + \pi} \} \phi^*_{v1x1} - \pi - (1-\phi^*_{v1x1})d ]
% % \end{align*}

% % \par If the underlined part of the above equation is positive, it directly follows from the above equation that $Eu_U[\phi^*_H = \phi^*_{v1x1}/p] - Eu_U[\phi^*_H = 0]$ is increasing in $\phi^*_{v1x1}$ (and decreasing in $d$). To make the underlined part positive, it must be the case that:

% % \begin{align*}
% % 	(1-\pi)R + 1 + \pi &\geq 0 \\
% % 	R &\geq - \frac{1 + \pi}{1-\pi}
% % \end{align*}

% \begin{align*}
% 	Eu_U[\phi^*_H = \phi^*_{v1x1}/p] &- Eu_U[\phi^*_H = 0] \\
% 	=& \kappa_a \cdot \kappa_a [ 2\phi^*_{v1x1}] + \\
% 	&\kappa_a \cdot \kappa_r [ 2 \pi \phi^*_{v1x1} ] + \\
% 	&\kappa_a \cdot (1-\kappa_a-\kappa_r) [ 2 \pi \phi^*_{v1x1} + (1-\pi)\phi^*_{v1x1}(A+1)] + \\
% 	&\kappa_r \cdot \kappa_a [ 2(1-\pi)\phi^*_{v1x1} ] + \\
% 	&\kappa_r \cdot \kappa_r [ 0 ] + \\
% 	&\kappa_r \cdot (1-\kappa_a-\kappa_r) [ (1-\pi)\phi^*_{v1x1}(R+1)] + \\
% 	&(1-\kappa_a-\kappa_r) \cdot \kappa_a [ \pi (-1) + 2\phi^*_{v1x1} - (1-\phi^*_{v1x1})d ] + \\
% 	&(1-\kappa_a-\kappa_r) \cdot \kappa_r [ \pi (2\phi^*_{v1x1}-1) - (1-\phi^*_{v1x1})d ] + \\
% 	&(1-\kappa_a-\kappa_r) \cdot (1-\kappa_a-\kappa_r) [ \pi (2\phi^*_{v1x1}-1) + (1-\pi)\phi^*_{v1x1} - (1-\phi^*_{v1x1})d ] \\
% 	= &(1 - \kappa_a - \kappa_r) \cdot \\
% 	&(\pi (2\phi^*_{v1x1} - 1) - (1-\phi^*_{v1x1})d + (1-\pi)\phi^*_{v1x1}(\underline{1+\kappa_a A + \kappa_r R})) + \\
% 	&2\kappa_a\phi^*_{v1x1}
% \end{align*}

% \par If the underlined part of the above equation is positive, it directly follows from the above equation that $Eu_U[\phi^*_H = \phi^*_{v1x1}/p] - Eu_U[\phi^*_H = 0]$ is decreasing in $d$ (given that $\phi^*_{v1x1}$ is decreasing in $d$). To make the underlined part positive, it must be the case that:

% \begin{align*}
% 	1 + \kappa_a A + \kappa_r R &\geq 0 \\
% 	R &\geq - (1+\kappa_a A)/\kappa_r
% \end{align*}

% \hfill $\blacksquare$

\clearpage
\subsection{Appendix C: The Alternative Game with the Status Quo Policy After the Abstention of the Pivotal Group}

\par Consider the alternative version of the voting game where the policy automatically stays at the status quo after the abstention of the pivotal group of voters. This game is equivalent to the game with one representative voter, randomly selected as uninformed or informed, with the probability $\pi$ of being uninformed. 

\par It should be cautioned that, in this alternative game, abstention and rejection vote of the pivotal group of voters induce the identical policy outcome. In either ways, the policy stays at the status quo. Abstention is different from rejection only because, after an abstention, a voter neither receives expressive benefit (d) nor pays voting cost (c). This is somewhat an unrealistic assumption compared to the original version of the game.

\noindent \textbf{Lemma C1}: \textit{Lemma 1 holds under the alternative version of the game.}\\ 

\noindent \textit{Proof}

\par In this proof, define $\Pi_{-g}$ as follows:
\begin{align*}
	\Pi_{-g} &= \begin{cases}
		\pi &\text{ if } g = I \\
		1-\pi &\text{ if } g = U
		\end{cases}
\end{align*} 

\noindent Then, the expected utilities for approval ideologues can be calculated as follows:

\begin{align*}
	EU_{\beta_g>1} (v_g=1, x_g=1) &= \Pi_{-g} (\beta_g + q) + (1-\Pi_{-g}) v_{-g} x_{-g} (\beta_g+q) + d - c \\
	&= (\Pi_{-g} + (1-\Pi_{-g})v_{-g} x_{-g}) (\beta_g+q) + d - c \\
	EU_{\beta_g>1} (v_g=1, x_g=0) &= (1-\Pi_{-g}) v_{-g} x_{-g} (\beta_g+q) - c \\
	EU_{\beta_g>1}(v_g=0) &= (1-\Pi_{-g})v_{-g} x_{-g} (\beta_g+q)
\end{align*}

\noindent Since $\beta_g+q>0$, It directly follows from the above utility functions that:

$$EU_{\beta_g>1} (v_g=1, x_g=1) > EU_{\beta_g>1} (v_g=1, x_g=0) > EU_{\beta_g>1}(v_g=0)$$

\par Similarly, the expected utilities for rejection ideologues can be calculated as follows:

\begin{align*}
	EU_{\beta_g<1} (v_g=1, x_g=1) &= \Pi_{-g} (\beta_g + q) + (1-\Pi_{-g}) v_{-g} x_{-g} (\beta_g+q) - c \\
	&= (\Pi_{-g} + (1-\Pi_{-g})v_{-g} x_{-g}) (\beta_g+q) - c \\
	EU_{\beta_g<1} (v_g=1, x_g=0) &= (1-\Pi_{-g}) v_{-g} x_{-g} (\beta_g+q) + d - c \\
	EU_{\beta_g<1}(v_g=0) &= (1-\Pi_{-g})v_{-g} x_{-g} (\beta_g+q)
\end{align*}

\noindent Since $\beta_g+q<0$, it directly follows from the above utility functions that:

$$EU_{\beta_g<1} (v_g=1, x_g=0) > EU_{\beta_g<1} (v_g=1, x_g=1) > EU_{\beta_g<1}(v_g=0)$$

\par From the above, it is shown that approval ideologues ($\beta_g>1$) and rejection ideologues ($\beta_g<-1$) always have an incentive to participate in the election and vote for their ideology. 

\hfill $\blacksquare$

\noindent \textbf{Lemma C2}: \textit{Lemma 2 holds under the alternative version of the game.}\\

\noindent \textit{Proof}

\par Suppose that informed voters are non-ideologues (i.e., $\beta_I \in [-1,1]$). Their utility functions can be written as follows:

\begin{align*}
	EU_I (v_I=1,x_I=1) &= (1-\pi) (\beta_I + q) + \pi v_U x_U (\beta_I + q) + d - c \\
	EU_I (v_I=1, x_I=0) &= \pi v_U x_U (\beta_I + q) + d - c \\
	EU_I (v_I=0) &= \pi v_U x_U (\beta_I + q)
\end{align*}

\noindent From the above, $EU_I (v_I=1,x_I=1) \geq EU_I (v_I=1, x_I=0)$ if and only if: 

$$(1-\pi) (\beta_I + q) \geq 0 \Rightarrow q = 1$$ 

\par From the above, $\beta_I + q > 0$ is always satisfied when non-ideologue informed voters are participating in the election. It follows directly from this fact that the below statement always holds:

$$max(EU_I (v_I=1,x_I=1), EU_I (v_I=1,x_I=0)) > EU_I (v_I=0)$$

\par Therefore, non-ideologue informed voters always have an incentive to participate in election and choose approval if $q=1$, choose rejection if $q=-1$.

\hfill $\blacksquare$

\noindent \textbf{Lemma C3}: \textit{In the voting game, the equilibrium stragety $\{v^*_U, x^*_U\}$  of non-ideologu uninformed voters can be represented by a threshold for the expected policy quality ($\phi$), or}\\

\begin{align*}
	x^*U &= \begin{cases}
		1 &\text{ if } \phi \geq \phi^*_x = \frac{1}{2} - \frac{\pi \beta_U}{2(\pi+d)}\\
		0 &\text{ otherwise}
	\end{cases}\\
	v^*_U &= \begin{cases}
		0 \text{ if } &\phi \geq \phi^*_{v1x0} = min\left\{ \phi^*_x, \phi^*_{vr} = \frac{d-c}{d} \right\}\\
		&\text{and }\\
		\phi \geq &\phi^*_{v1x0} = max\left\{ \phi^*_x, \phi^*_{va} = \frac{\pi(1-\beta_U) + c}{2\pi + d} \right\}
	\end{cases}
\end{align*}

\noindent \textit{Proof}

\par The utility functions of non-ideologue uninformed voters (i.e., $\beta_U \in [-1,1]$) can be calculated as follows:

\begin{align*}
	EU_U (v_U=1,x_U=1) = &\pi ( \phi(\beta_U+1) + (1-\phi)(\beta_U-1)) + \\
	&(1-\pi)(\phi(1-\kappa_r)(\beta_U+1) + (1-\phi)\kappa_a(\beta_U-1) ) + \\
	&\phi d-c \\
	EU_U (v_U=1,x_U=0) = &(1-\pi) (\phi(1-\kappa_r)(\beta_U+1) + (1-\phi)\kappa_a(\beta_U-1) ) + \\
	&(1-\phi) d-c \\
	EU_U (v_U=1) = &(1-\pi) (\phi(1-\kappa_r)(\beta_U+1) + (1-\phi)\kappa_a
\end{align*}

\noindent From the above, $EU_U (v_U=1,x_U=1) \geq EU_U (v_U=1,x_U=0)$ if and only if:

\begin{align*}
	\pi ( \phi(\beta_U+1) + (1-\phi)(\beta_U-1)) &\geq d (1-2\phi) \\
	\phi &\geq \frac{1}{2} - \frac{\pi \beta_U}{2(\pi+d)} = \phi^*_x
\end{align*}

\par If $\phi \geq \phi^*_x$, $EU_U (v_U=1,x_U=1) \geq EU_U (v_U=0)$ if and only if:

\begin{align*}
	\pi ( \phi(\beta_U+1) + (1-\phi)(\beta_U-1)) \phi d - c&\geq 0 \\
	\phi(2\pi+d) &\geq \pi(1-\beta_U) + c \\
	\phi &\geq \frac{\pi(1-\beta_U) + c}{2\pi+d} = \phi^*_{va}
\end{align*}

\par If $\phi < \phi^*_x$, $EU_U (v_U=1,x_U=1) > EU_U (v_U=0)$ if and only if:

\begin{align*}
	(1-\phi)d - c &> 0 \\
	\phi^*_{vr} &< \frac{d-c}{d}
\end{align*}

\hfill $\blacksquare$

\noindent \textbf{Proposition C1}: \textit{In the voting game, the lower-bound of the abstention interval ($\phi^*_{v1x0}$) is independent of $\pi$. The upper bound of the abstention interval ($\phi^*_{v1x1}$) is weakly increasing in $\pi$ if and only if the ideology is sufficiently leaning to approval (i.e., higher $\beta_U$) and the ratio of the expressive benefit to voting cost ($d/c$) is sufficiently high, or}\\

$$\frac{d}{c} \geq \frac{2}{1-\beta_U}$$

\noindent \textit{Proof}

\par Consider $\phi^*_{vr}$. It directly follows from the function that $\phi^*_{vr}$ is independent of $\pi$. Consider $\phi^*_{va}$. Taking a partial derivative of the function with respect to $\pi$:

\begin{align*}
	\frac{\partial \phi^*_{va}}{\partial \pi} &= \frac{-2c - \beta_U d + d}{(2\pi + d)^2} \\
	&\geq 0 \text{ if and only if }\\
    \frac{d}{c} \geq \frac{2}{1-\beta_U} 
\end{align*}

\hfill $\blacksquare$

\par From Proposition C1, two critical differences emerge between the alternative model and the original model in th paper. First, there is no mixed abstention context in the alternative game because only the upper bound of the abstention interval is endogenous to $\pi$. The uninformed pivot probability can influence whether the voter deviates from/to abstention to/from approval, but cannot influence the deviation from/to abstention to/from rejection. Second, the probability of informed ideologues ($\kappa_a$, $\kappa_b$)plays no role in differentiating the abstention context. Therefore, the result that the abstention is increasing in $\pi$ is not necessarily consistent with the implication from Feddersen and Pessendorfer (1996). Uninformed voters are not exactly ``delegating'' their voters to informed voters.

\par In the alternative extended accountability game, Lemma 5 continues to hold, because this lemma is not endogenous to the action of voters. Also, Proposition 4 continues to hold because this proposition is only dependent upon the actions of ideologues and non-ideologue informed voters, which are identical in the original and the alternative version of the game (Lemma C1 and Lemma C2). 

\par Then, the following proposition holds in the alternative version of the accountability game:

\noindent \textbf{Proposition C2}: \textit{Suppose that $\phi^*_H[\pi=0]=0$ holds (i.e., $\kappa_{a}+\kappa_{r} \geq 0.5$). If $\pi>0$ (i.e., uninformed voters are potentially pivotal), $\phi^*_H$ deviates to $\phi^*_{v1x1}/p$ for a:
\begin{itemize}
	\item sufficiently high $\pi$ with
	\item sufficiently high $p$,  
	\item sufficiently low  $\kappa_{a}+\kappa_{r}$,  
	\item sufficient high $d$, and 
	\item sufficiently low $c$
\end{itemize}
}

noindent \textit{Proof}

\par To start with, the rejection vote and the abstention of the pivotal group of voters induce the same result (i.e., status quo policy) for the policymaker. Therefore, following statement always holds:

$$EU_{P_H}[\phi_H = \phi^*_{v1x0}/p, v_U = 1, x_U = 0] \leq EU_{P_H}[\phi_H=0, v_U=1, x_U=0]$$

\par Then, if $p \geq \phi^*_{v1x1}$, $P_H$ prefers $\phi_H = \phi^*_{v1x1}/p$ over $\phi_H = 0$ if and only if:
\begin{align*}
EU_{P_H}[\phi_H = \phi^*_{v1x1}/p, v_U = 1, x_U = 1] &> EU_{P_H}[\phi_H=0, v_U=1, x_U=0] = 2 \kappa_a \\ 
\pi \left(1-\frac{\phi^*_{v1x1}}{p}\right)(1-\kappa_a-\kappa_r) &> \frac{\phi^*_{v1x1}}{p} (\kappa_a + \kappa_r - 0.5) \\
\pi (1-\kappa_a-\kappa_r) &> \frac{\phi^*_{v1x1}}{p} (\pi(1-\kappa_a-\kappa_r) + (\kappa_a + \kappa_r - 0.5)) \\
\end{align*}
\noindent The above inequality can be rewritten by using the function $\Theta$. The inequality holds if and only if $\Theta$ is larger than zero:
\begin{align*}
0 <& \pi (1-\kappa_a-\kappa_r) - \frac{\phi^*_{v1x1}}{p} (\pi(1-\kappa_a-\kappa_r) + (\kappa_a + \kappa_r - 0.5)) \\
0 <& \pi( (1-\kappa_a-\kappa_r)(\pi (2p-1) + pd - c) - (\kappa_a+\kappa_r-0.5)) \\ 
&- (\kappa_a+\kappa_r-0.5)c = \Theta 
\end{align*}
\noindent By assumption, $p\in[0,1]$, $1-\kappa_a-\kappa_r > 0$, $\pi \in (0,1]$, and $\kappa_a+\kappa_r \geq 0.5$. Therefore, $\Theta$ is increasing in $p$ and $d$ and decreasing in $c$ and $\kappa_a+\kappa_r$. Also, the inequality $\Theta>0$ holds only when $\Theta$ is increasing in $\pi$ at $\pi=0$. 

\par If the abstention interval does not exist for non-ideologue uninformed voters (i.e., $\phi^*_{v1x0}=\phi^*_{v1x1}=\phi^*_x=0.5$) and $p \geq 0.5$, $P_H$ prefers $\phi_H = \phi^*_{v1x1}/p$ over $\phi_H = 0$ if and only if:
\begin{align*}
EU_{P_H}[\phi_H = 1/2p, v_U = 1, x_U = 1] &> EU_{P_H}[\phi_H=0, v_U=1, x_U=0] = 2 \kappa_a \\ 
%\pi \left(1-\frac{1}{2p}\right)(1-\kappa_a-\kappa_r) &> \frac{1}{2p} (\kappa_a + \kappa_r - 0.5) \\
\pi (1-\kappa_a-\kappa_r) &> \frac{1}{2p} (\pi(1-\kappa_a-\kappa_r) + (\kappa_a + \kappa_r - 0.5)) \\
p &> \frac{1}{2}+ \frac{\kappa_a + \kappa_r - 0.5}{2\pi(1-\kappa_a-\kappa_r)}
\end{align*}
\noindent The above function implies that the inequality holds for sufficiently high $p$ and $\pi$ and sufficiently low $K = \kappa_a + \kappa_r$.

\hfill $\blacksquare$

\par From Proposition C2, if $\kappa_a + \kappa_r \geq 0.5$, accountability improvement at $\phi^*_{v1x1}$ in response to higher $\pi$ occurs under the similar condition as Proposition 5. However, accountability improvement never occurs at $\phi^*_{v1x0}$, and thus with lower $d/c$ ratio. This is due to the fact that rejection and abstention induce the same policy outcome, the status quo policy. 

\clearpage
\subsection{Appendix D: The Prevalence of the Accountability Improvement (Extension)}

\renewcommand{\thefigure}{A\arabic{figure}}
\setcounter{figure}{0}

\begin{figure}[ht!]
	\caption{The Prevalence of the Accountability Improvement in Response to Uninformed Voting (The Smaller Voting Cost)}
	\label{fig:rangegraph2}
	\includegraphics[width=\linewidth]{figure/rangegraph-2}
\end{figure}

\begin{figure}[ht!]
	\caption{The Prevalence of the Accountability Improvement in Response to Uninformed Voting (The Larger Voting Cost)}
	\label{fig:rangegraph3}
	\includegraphics[width=\linewidth]{figure/rangegraph-3}
\end{figure}

\begin{figure}[ht!]
	\caption{The Prevalence of the Accountability Improvement in Response to Uninformed Voting (Unequal Probabilities of Approval and Rejection Ideologues)}
	\label{fig:rangegraph4}
	\includegraphics[width=\linewidth]{figure/rangegraph-4}
\end{figure}

\clearpage
\subsection{Appendix E: Simulation Codes for Comparative Statics}

\par \nolinkurl{Kato2019thlo_simulations.R} contains the R codes to replicate Figure 2, 3, 4, 5, and other Appendix figures. 

%TC:endignore

%\clearpage
%\pagenumbering{arabic}
%\setcounter{page}{35}